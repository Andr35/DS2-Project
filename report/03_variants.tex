\section{Variants}
\label{sec:variants}

We propose two different variants to the original protocol.
The first one is related to the way the nodes gossip with each other.
The second one is about how nodes chooses their peers.
We discuss the effect of these changes in \cref{sec:experiments}.


\subsection{Push-Pull}
The protocol described in the original paper uses the Pull approach.
In other words, a node selects its peer and sends it a message with some information.
The peer receives the message but does not reply.

The Push-Pull approach introduces an additional step.
The selected peer receives the message, updates its local knowledge about the systems and replies to the sender with its up-to-date information.
The sender receives the reply and updates its own one.
The idea behind Push-Pull is to spread the information as fast as possible.


\subsection{Peer Selection Strategy}
In the basic protocol, each node selects its peer randomly at each round of gossip.
We propose some different peer selection strategies that try to exploit the local knowledge to make a better choice.

\paragraph*{Quiescence}
The probability to select a node is linear with respect to its quiescence, i.e. the difference (in gossip intervals) between the current time and the last heartbeat update received for that node.

\paragraph*{Squared Quiescence}
The probability to select a node is proportional to the square of its quiescence.

\paragraph*{Highest Quiescence}
Always select the node with the oldest update.
If more than one node share the oldest update time, select one randomly.
